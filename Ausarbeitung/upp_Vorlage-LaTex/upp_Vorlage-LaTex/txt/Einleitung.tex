%:::::::::::::::::::::::::::::::::::::::::::::::::::::::::::::
\chapter{Einleitung}
\label{cptr:Einleitung}
%:::::::::::::::::::::::::::::::::::::::::::::::::::::::::::::
Vorwörter sind so ähnlich wie Bedienungsanleitungen – meistens werden sie nicht gelesen, besonders von Technikern. Trotzdem wollen wir es nutzen, um die Hintergründe und die Zielsetzung dieses Buches zu erläutern. Eigentlich wollten wir zwei Bücher schreiben. Eines für die Studierenden mit Grundlagen und einigen Beispielen und das zweite für die Praktiker mit wenig Einführung und vielen Beispielen. Wie Sie sehen, haben wir uns anders entschieden. Der Grund resultiert aus Erfahrungen mit unserer Lehrveranstaltung „Messen von Stoff- und Energieströmen“. Dort haben wir Studierende und Unternehmensvertreter in Weiterbildungsangeboten gemischt. Das Ergebnis war ermutigend. Die Studierenden profitierten von den vielen Beispielen der Praktiker. Diese wiederum fanden (wieder) Gefallen am studentischen Lernen. Das Buch ist so geschrieben, dass jedes Kapitel, ja nahezu jeder Abschnitt, alleinstehend gelesen werden kann. Insofern schauen Sie ins Inhaltsverzeichnis und lesen Sie (zuerst), was Sie am meisten interessiert. Erwartungsgemäß werden sich die Praktiker zunächst auf Kapitel \ref{cptr:Kapitel4} und \ref{cptr:Kapitel5} konzentrieren. Wir haben dort jedoch Verweise auf die Grundlagen Kapitel \ref{cptr:Kapitel1} bis \ref{cptr:Kapitel3} hinterlegt. Insgeheim hoffen wir, dass sie doch irgendwann die Neugier packt, wie wir zu den dort aufgeführten Ergebnissen gekommen sind. Dann haben wir unser Ziel erreicht. Dies gilt umgekehrt auch für die Studenten, die hoffentlich neugierig genug werden, was sie mit dem Wissen aus Kapitel \ref{cptr:Kapitel1} bis \ref{cptr:Kapitel3} denn nun anfangen können.\par\bigskip

Sie finden in dem Buch nicht nur Wissen und Erfahrungen aus unserer eigenen Arbeit. Es gibt eine Vielzahl von fleißigen Kollegen und Unternehmensvertretern, die ebenfalls spannende Ergebnisse erarbeitet haben. An dieser Stelle herzlichen Dank an diese für die Veröffentlichung der Ergebnisse. Wir geben auch gerne zu, dass die Grundlagen zu Thermodynamik und Wärmeübertragung in Kapitel \ref{cptr:Kapitel1} nicht wir erarbeitet haben. Allerdings haben wir einige Anwendungsbeispiele integriert, damit es nicht ganz so trocken ist. Ergänzend haben wir uns bemüht, die Inhalte durch einen etwas lockeren Sprachstil nicht zu akademisch zu vermitteln. Die Hardliner der Wissenschaft mögen es uns daher verzeihen, dass die Wortwahl nicht immer einem streng universitären Anspruch genügt.\par\bigskip

Das Buch entstand als Teamarbeit mit den Mitarbeitern meines Fachgebietes an der Universität Kassel und Vertretern der Limón GmbH als spin-off-Unternehmen. Die Grafiken und die Textgestaltung erstellten die fleißigen Designerinnen von formkonfekt. Mein Dank geht auch an das Lektorat Maschinenbau im Vieweg + Teubner Verlag, das kompetent, engagiert und mit vielen Anregungen den Buchentstehungsprozess begleitet hat. Ohne die Unterstützung aller genannten Beteiligten wäre das Buch heute noch nicht fertig. Und ohne die finanzielle Unterstützung des NATURpur Institutes der HSE AG aus Darmstadt wären wir auch nicht in der Lage gewesen, das Material so schick aufzuarbeiten. Allen sei an dieser Stelle nochmals herzlich gedankt.\par\bigskip

Denjenigen, die dieses Vorwort doch gelesen haben, wünschen wir nun viel Spaß beim Lesen und anschließend viel Erfolg bei der Anwendung – sei es in der Lehre oder im Beruf \parencite[S.~V]{Hesselbach.2012}.