\chapter{Kapitel1}
\label{cptr:Kapitel1}
%:::::::::::::::::::::::::::::::::::::::::::::::::::::::::::::
blablablabla
%:::::::::::::::::::::::::::::::::::::::::::::::::::::::::::::
\section{Lippenbekenntnis oder Zukunftsmodell}
\label{sctn:Lippenbekenntnis oder Zukunftsmodell}
%:::::::::::::::::::::::::::::::::::::::::::::::::::::::::::::

Mit der Ratifizierung des Kyoto-Protokolls hat sich die Bundesregierung verpflichtet, die Treibhausgasemissionen test
bis 2020 um 40\% gegenüber 1990 zu vermindern. Dieses Klimaschutzziel ist
nur zu erreichen, wenn neben dem Ausbau der Erneuerbaren Energien auch die Energieeffizienz
in allen Bereichen deutlich gesteigert wird. So soll beispielsweise die Energieproduktivität
in Deutschland nach dem Willen der Politik im Vergleich zu 1990 bis zum Jahr 2020 verdoppelt
werden, d. h. im Klartext: Es steht nur noch die Hälfte an Energie für die Herstellung der gleichen
Produkte zur Verfügung. Trotz dieser ambitionierten Zielsetzung erscheinen die Maßnahmen
zu deren Unterstützung im Vergleich zur Förderung der Erneuerbaren Energien über das
EEG eher dürftig und unstrukturiert. Es ist daher nicht weiter verwunderlich, dass die Realität
dem gesteckten Ziel deutlich hinterher hinkt. Statt der jährlich notwendigen 3 \% Steigerung
erreichen wir aktuell noch nicht einmal mehr die Hälfte, im summarischen Mittel seit 1990
ca. 1,7 \% (siehe Abbildung \ref{fig:1}).\\

\begin{figure}[h]					%h (here) - Gleicher Ort
													%t (top) - Oben auf der Seite
													%b (bottom) - Unten auf der Seite
													%p (page) - Auf einer eigenen Seite
													%! (override) - Erzwingt die angegebene Position
	\centering
		\missingfigure{}			% Platzhalter
	\caption{xxx}
	\label{fig:1}
\end{figure}

Die Gründe sind vielschichtig und in den einzelnen Bereichen der Gesellschaft auch unterschiedlich. Betrachtet man Effizienzmaßnahmen unter Marketinggesichtspunkten, so ist das Thema umgangssprachlich formuliert leider absolut unsexy. Effizienz klingt nach weniger, riecht also nach Verzicht, schmeckt wie Styropor und ist meist einfach unsichtbar. Da bieten die Erneuerbaren Energien deutlich mehr fotogenes Potenzial für Politiker, um sich zu präsentieren. Zudem fällt auch eine politische Polarisierung und damit Differenzierung deutlich leichter. Das schon historische Spannungsfeld „Atomkraft gegen Windkraft“ verdeutlicht dies eindrucksvoll, wobei Letztere bei Hardlinern auch gerne als „Vogelschredderanlagen“ bezeichnet werden.

Diese Möglichkeiten für die Öffentlichkeitsarbeit bieten Effizienzmaßnahmen nicht. Ihre Notwendigkeit ist zudem über alle Parteien hinweg weitgehend unumstritten. Auf den ersten Blick erscheint eine solch breite Mehrheit im Hinblick auf eine langfristige und kontinuierliche politische Unterstützung von Vorteil. Faktisch führt dies jedoch eher dazu, dass der Energieeffizienz nicht die Aufmerksamkeit und Unterstützung zuteil wird, die ihrem Anteil zum Erreichen der Klimaschutzziele entspricht. Wo ist das Pendant zum EEG?

Ungeachtet dieses politischen Nachteils gibt es eine Reihe von weiteren Umständen, welche die Umsetzung von Energieeffizienzmaßnahmen behindern oder verzögern. Dabei muss allerdings zwischen den einzelnen gesellschaftlichen Bereichen differenziert werden (siehe Abbildung \ref{fig:2}).\\

\begin{figure}[h]
	\centering
		\missingfigure{}			% Platzhalter
	\caption{xxx}
	\label{fig:2}
\end{figure}
