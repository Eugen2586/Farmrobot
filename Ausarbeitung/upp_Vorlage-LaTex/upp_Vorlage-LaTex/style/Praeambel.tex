%::: Koma-Script Buch Klasse
\documentclass[
DIV=14, 						% Wert zur Blattaufteilung bei der Berechnung des Satzspiegels
BCOR=15mm, 					% Bindekorrektur
fontsize=12pt, 			% Schriftgröße 
a4paper, 						% Blattgröße
oneside,						% Einseitiges Dokument, für zweiseitiges Dokument (twoside)						
openany, 						% Verhindert leere Seiten nach \input{} Befehl
toc=listof,					% Abbildungs- und Tabellenverzeichnis ins Inhaltsverzeichnis
toc=bibliography		% Literaturverzeichnis (Schriftum) ins Inhaltsverzeichnis
]
{scrbook}

%::: Anpassung an deutsche Regeln
\usepackage[ngerman]{babel}
\usepackage[T1]{fontenc}
\usepackage[utf8]{inputenc} % UTF Codierung, Vorraussetzung: alle Dateien als utf 8 abspeichern

%::: Darstellung aller Verzeichnisse
\usepackage{morewrites} % erlaubt mehr Streams, löst Fehler "No room for a new \write"

%::: Einbindung von Sonderzeichen und Symbolen
\usepackage{eurosym} 														% Eurosymbol
\usepackage{textcomp} 													% Symbole wie copyright, bullet, yen

%::: Textsatzbibliotheken & Einstellungen
\usepackage{lmodern} 														% Schriftart, da das andere Paket probleme macht
\usepackage{a4wide}															% Einstellen der Seitenränder
\usepackage[onehalfspacing]{setspace}						% Bereiche mit anderen Zeilenabstände definieren
\parindent 0pt																	% Kein Einrücken nach Absatz

%::: Bearbeiten der Kopf und Fußzeile
\usepackage{scrpage2}														% KOMA-Script Ergänzungspaket für Kopf- und Fußzeilen
\pagestyle{scrheadings}													% KOMA-Script Ergänzungspaket für Kopf- und Fußzeilen
\setheadsepline{.4pt}														% Linie in der Kopfzeile
\setlength{\headheight}{1.1\baselineskip}
\addtokomafont{captionlabel}{\bfseries} 				% caption labels fett drucken
\setcapindent{0em} 															% kein Einzug bei captions
\cfoot[]{}																			% Fußzeile Seitennummer rechts unten, Abschalten bei zweisetigem Druck
\ofoot[\pagemark]{\pagemark}
\newcounter{savepage}														% Seitenzahlen speichern
\chead[]{}
\ohead{\headmark}																% Kopfzeile
\automark[chapter]{chapter}

%::: Grafikbibliotheken
\usepackage{graphicx} 					% Zum Laden von Grafiken
\usepackage{subfig} 						% Zwei Abbildungen lassen sich nebeneinander darstellen
\usepackage{here} 							% Erzwingt feste Grafikposition mit der Angabe eines großen H: \begin{figure}{H}
\pdfoptionpdfminorversion=6     % Vermeidet Warnungen zu PDF Versionen beim Einbinden von Grafiken
\usepackage{pdfpages}

%::: Tabellenbibliotheken 
\usepackage{tabularx}						% Tabellen mit definierter Breite
\usepackage{multirow}						% Zellen zusammenfassen
\usepackage{longtable}					% Für lange Tabellen auch über mehrere Seiten
\usepackage{booktabs}						% Für Linien in Tabellen z. B. \toprule
\usepackage{colortbl,hhline}
	\definecolor{upp}{rgb}{0.584, 0.757, 0.129}		% Definiere eigene Farbe, hier "upp grün"'

%::: Mathematik- und Phyisikbibliotheken
\usepackage{amsmath}
\usepackage{amsfonts}
\usepackage{amssymb}
\usepackage{physics}
\usepackage[thinspace, thinqspace, squaren, textstyle]{SIunits} % Einheiten, ggf. hier in Zukunft zu Paket siuntix wechslen

%::: Aufzählungen
\usepackage{paralist}						% Macht schönere Listen (geringere Abstand zwischen zwei Punkten)

%::: Zitations- und Referenzierungsbibliotheken
\usepackage[natbib=true, style=authoryear, backend=biber, maxcitenames=1 , maxbibnames=99, uniquename=false, uniquelist=false, sortcites = true, dashed=false]{biblatex}
\renewcommand*{\nameyeardelim}{\addspace} % Kein Komma zwischen Autor und Jahr in der Ziation
\addbibresource{Vorlage_xxx.bib}									% HIER die BIBLIOTHEKSDATEI EINFÜGEN
\DefineBibliographyStrings{ngerman}{ 			% Beschriftung in Tags mit et al. statt u.a.
   andothers = {{et\,al\adddot}},            
}
\usepackage[german=quotes]{csquotes}
\renewcommand*{\finalnamedelim}{,\space} 	% Trennzeichen zwischen den Autoren im Literaturverzeichnis
\usepackage{enumitem}											% Für individuelle Einstellungen der Umgebungen enumerate, itemize und description
\usepackage{url}													% Für die Angabe von URLs
\usepackage{chngcntr} 										% Durchnummerierung der Fußnoten
\counterwithout{footnote}{chapter}				% Durchnummerierung der Fußnoten

\usepackage[															% Verlinkungen im Dokument aktivieren
	bookmarks=true,
	plainpages=false,
	pdfpagelabels,
	hypertexnames=false,
	pdfborder={0 0 0}, 			% Keine roten Rahmenlinien
	colorlinks=false, 			% Allgemeine Einstellung zu Farbe
	urlcolor=black, 				% Farbe der URL Links zu Web oder Mail
	linkcolor=black,  			% Farbe der allgemeinen Links
	citecolor=black] 				% Farbe für Quelllinks aus dem Schriftum
{hyperref} 
\usepackage{url}
\newcommand*\oldurlbreaks{} 					% bricht lange urls an der richtigen Stelle
\let\oldurlbreaks=\UrlBreaks
\renewcommand{\UrlBreaks}{\oldurlbreaks\do\a\do\b\do\c\do\d\do\e%
  \do\f\do\g\do\h\do\i\do\j\do\k\do\l\do\m\do\n\do\o\do\p\do\q%
  \do\r\do\s\do\t\do\u\do\v\do\w\do\x\do\y\do\z\do\?\do\&}
\hypersetup{ % Meta Informationen für PDF Datei
  pdftitle={\thesistitle}, %%
  pdfauthor={\vornamenachname}, %%
  pdfsubject={\thesistype arbeit}, %%
  pdfcreator={\vornamenachname}, %% 
  pdfproducer={}, %%
  pdfkeywords={Universität Kassel, Fachgebiet Umweltgerechte Produkte und Prozesse}, %%
	}

\usepackage{todonotes} 		% Kommentare für Korrekturen
\usepackage{currvita} 		% Paket für Lebenslauf

\usepackage[nonumberlist,acronym,section]{glossaries}
\newglossary[slg]{symbolslist}{syi}{syg}{Symbolverzeichnis} 	%Ein eigenes Symbolverzeichnis erstellen
\newglossary[lsg]{Latein}{lsi}{lso}{Lateinische Symbole} 			%Ein eigenes Symbolverzeichnis erstellen
\newglossary[gsg]{Griechisch}{gsi}{gso}{Griechische Symbole} 	%Ein eigenes Symbolverzeichnis erstellen
\newglossary[ilg]{indizeslist}{ini}{ing}{Indizeverzeichnis} 	%Ein eigenes Indizeverzeichnis erstellen
\renewcommand*{\glspostdescription}{} 					%Den Punkt am Ende jeder Beschreibung deaktivieren

%::: STYLE FÜR TABELlE OHNE EINHEITEN (ABKÜRZUNGEN UND INDIZES)
\newglossarystyle{Einfach}{ % Definition eines neuen Styles
  \renewenvironment{theglossary}% 
    {\begin{longtable}[l]{@{}lp{10cm}}}% [l]= linksbündige longtable, %@{}= kein Abstand vor erster Spalte 
    {\end{longtable}}%
 \renewcommand*{\glossaryheader}{%
    \bfseries Symbol&\bfseries\descriptionname\tabularnewline\endhead
		}% 
		\renewcommand*{\glsgroupheading}[1]{}% 
  \renewcommand{\glossentry}[2]{% 
    \glsentryitem{##1}\glstarget{##1}{\glossentryname{##1}} & \glossentrydesc{##1}\tabularnewline 
  }% 
  \renewcommand*{\glsgroupskip}{% 
    \ifglsnogroupskip\else &\tabularnewline\fi
		}% 
}
%::: STYLE FÜR TABELlE MIT EINHEITEN (LATEINISCHE UND GRIECHISCHE ZEICHEN)
\newglossarystyle{Tabelle}{% 
  \renewenvironment{theglossary}% 
    {\begin{longtable}[l]{@{}lp{10cm}l}}% [l]= linksbündige longtable, %@{}= kein Abstand vor erster Spalte 
    {\end{longtable}}% 
  \renewcommand*{\glossaryheader}{% 
    \bfseries Symbol&\bfseries\descriptionname& 
    \bfseries Einheit\tabularnewline\endhead
		}% 
  \renewcommand*{\glsgroupheading}[1]{}% 
  \renewcommand{\glossentry}[2]{% 
    \glsentryitem{##1}\glstarget{##1}{\glossentryname{##1}} & 
    \glossentrydesc{##1} & \glossentrysymbol{##1}\tabularnewline 
  }% 
  \renewcommand{\subglossentry}[3]{% evtl noch anpassen 
     & 
     \glssubentryitem{##2}% 
     \glstarget{##2}{\strut}\glossentrydesc{##2} & 
     ##3\tabularnewline 
  }% 
  \renewcommand*{\glsgroupskip}{% 
    \ifglsnogroupskip\else & &\tabularnewline\fi
		}% 
} 

\makeglossaries