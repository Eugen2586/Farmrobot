%----------------------------------------------
%Präambel
\documentclass[a4paper,12pt,titlepage]{article}
\usepackage{ngerman}
%\usepackage[latin1]{inputenc}
\usepackage[utf8]{inputenc}
\pagestyle{headings}
\usepackage{blindtext} %Zusatzpackage Blindtext
\usepackage{amsmath} %mathemaitscher Formelsatz
\usepackage{amssymb} %mathematische Symbole
\hyphenation{Stau-becken} %immer diese Trennung
\usepackage{verbatim}%Text im Original setzen (ohne Deutung)
%\usepackage{comment}
\usepackage{eurosym} 
\usepackage{textcomp} %Textkomponenten
\usepackage{hhline} %Tabellenmodifikation
\usepackage{graphicx}
\usepackage{makeidx} %Stichwortverzeichnis
\usepackage{colortbl} %Tabellenfarbmanipulation
\usepackage{lscape} %Hoch und Querformat Wechsel


%\parskip 2.1ex %höhenabstand
%\parindent 1.9 em %breitenabstand
\makeindex % ab hier nach Stichwörtern suchen
%----------------------------------------------

\begin{document}
 %Titelseite
\title{{\textsf{Projektarbeit\\
	\glqq Automatisierung von Prozessen der Agrarwirtschaft mittels roboterisierten Werkzeugen\grqq \\
	\vspace{1ex}
	Dozent: Peter Zipf}}}
\author{Christian Küllmer, Jonas Kallweidt\\
	35202882}
\date{Kassel, den \today}

%Verzeichnisse
\maketitle       %Titelseite
\tableofcontents %Inhaltsverzeichnis
\listoffigures   %Abbildungsverzeichnis
\listoftables    %Tabellenverzeichnis

!TEX root = Praktikumsarbeit.tex  %zur fehlerfreien Compilierung und Einbettung

\section{Einleitungs}
Einleitungstext, worum geht es hier bei diesem Projekt?

\subsection{Motivation} 


Die Motivation sich mit dem Thema der Automatisierung im Land und Gartenbau zu beschäftigen, kann wohl jeder verstehen, dem die Zeit fehlt, regelmäßig seinen Garten zu pflegen. Fehlte dem Besitzer einige Wochen lang, die Muße sich um seinen Garten zu kümmern, sind die Pflanzen im Garten zusammen mit dem Unkraut vertrocknet. An Fruchternte oder gar Zucht ist dann nicht mehr zu denken. Die Zeit der Arbeitskraft, die ein Garten bindet, ist gerade im Stadium des Pflanzenwachstums immens.
\newline
Die bisherigen Ansätze zur Automatisierung der Landwirtschaft beschäftigten sich zumeist nicht damit, die Zeit des Arbeiters auf dem Feld zu reduzieren, sondern lediglich die zu bearbeitende Fläche zu erhöhen. So ermöglicht der Einsatz des Pflugs gegenüber dem Spaten ein vielfaches der Fläche in einem Arbeitstag zu bearbeiten. 
\newline 
Die menschliche Arbeitskraft hat für uns dabei noch mehr Nachteile, als den bloßen Faktor Mensch. Ein Mensch kann nicht für gleichbleibende Feuchtigkeit im Boden sorgen, sondern sorgt immer für eine gleichbleibende Nachgießmenge. Wir wollen also in der Zeit der knappen Wasser und Ackerflächen und anschauen, ob es in diesem Bereich Optimierungspotentiale gibt und wie man diese mit einer Maschine nutzbar machen kann. Damit können wir einen Beitrag leisten, den Wasserverbrauch von frischen Gemüse zu reduzieren. Dabei gilt es, dass unsere Maschine Daten aufnehmen soll um damit Auswertungen die äußeren Bedingungen der Pflanzen fahren zu können. 

Dabei geht es also darum, die wiederkehrenden Tätigkeiten der Gartenarbeit zu verringern und die Auswertbarkeit weiter in den Mittelpunkt zu stellen. Ziel dabei sollen die Verrinerung von Ressourcen und Arbeitstätigkeit je Gemüseeinheit werden.
\newline
Dabei wollen wir einen Pflanzenumgang der möglichst ohne das Einbringen von Pestizieden oder Fungizieden auskommt.

\subsection{aktueller Stand}
kurze Bestandsmeldung der agrarwirtschaftlichen Methoden

\subsection{Projektbeschreibung}
auf welche Aspekte aus der Bestandsmeldung wird hier weiterführend eingegangen

%!TEX root = Praktikumsarbeit.tex  %zur fehlerfreien Compilierung und Einbettung

\section{Aufbau und Funktionsweise}
\subsection{Allgemeiner Aufbau}
Grundkonzept,
Karthesische Koordinaten
Systematische Beschreibung
\subsection{Aufbau des Bewegungsapparats}
Logische Komponenten


\end{document}