%Praeambel
%----------------------------------------------
\documentclass[a4paper,12pt,titlepage]{article} %article, beamer, book, report, scrartcl, scrbook, scrreprt, slides
\usepackage[utf8]{inputenc} %Zeichensatz waehlen
%\usepackage[latin1]{inputenc}
\usepackage{amsmath} %mathemaitscher Formelsatz
\usepackage{amssymb} %mathematische Symbole
\usepackage{ngerman} % Schriftsprache
\usepackage{blindtext} %Blindtexgenerierung
\usepackage{eurosym} %Eurozeichen
\usepackage{textcomp} %Textkomponenten
\usepackage{graphicx} %Bildeinbettung
\usepackage{url} %belaesst Leerzeichen in Text
\usepackage{verbatim} %Text im Original setzen (ohne Deutung)
\usepackage{makeidx} %Stichwortverzeichnis
\usepackage{color} %Farben
\usepackage{colortbl} %Tabellenfarbmanipulation
\usepackage{booktabs} %Tabellen, horizontale Linien
\usepackage{multicol} %mehrere Spalten im Text
\usepackage{lscape} %Hoch und Querformat Wechsel
%----------------------------------------------
\makeindex %nach Stichworten suchen,spaeter dann \printindex
\RequirePackage{hhline} %Doppellinie in Tabelle
%\pagestyle{headings} %Kapitel und Seitennummer am Rand
\hyphenation{Stau-becken} %immer diese Trennung nutzen
%----------------------------------------------
%Absatzmanipulation
\parskip 1ex %Höhenabstand in ex oder cm (standard 0ex)
\parindent 0em %Breitenabstand in em oder cm (standard 1em)
%verhindern mit \noindent
% Tabellenmanipulation
\newcolumntype{G}[1]{>{\small \columncolor[gray]{0.8}}p{#1}}
\newcolumntype{P}[1]{>{\small \hspace{0pt}}p{#1}}



\begin{document}
%Titelseite
\title{{\textsf{Meine Vorbereitung auf die\\ 
\glqq Einf"uhrung in das wissenschaftliche Arbeiten mit dem Textsatzprogramm \LaTeX\grqq \\
\vspace{1ex}
bei Jochen Wulfhorst}}}
\author{Jonas Kallweidt}
\date{Kassel, den \today}
\maketitle       %Titelseite
\tableofcontents %Inhaltsverzeichnis
\listoffigures   %Abbildungsverzeichnis
\listoftables    %Tabellenverzeichnis

\section{Formatierungen}
\subsection{Schriftarten}
\textbf{erstes Wort} Fettdruck\\
\textsc{erstes Wort} Kapitälchen\\
\textit{erstes Wort} Kursiv\\
\textsf{erstes Wort} serifenlos\\
\texttt{erstes Wort} Schreibmaschine\\

\begin{bfseries}
 fetter Text \\
\end{bfseries}
\begin{itshape}
 kursiver Text \\
\end{itshape}

\subsection{Gr"o"sen}
{\tiny winzig} \\
{\scriptsize sehrklein} \\
{\footnotesize fussnotengr"osse} \\
{\small klein} \\
{\normalsize normal} \\
{\large gross} \\
{\Large gr"osser} \\
{\LARGE nochgr"osser} \\
{\huge riesig} \\

\subsection{Aufz"ahlungen}
Vorbereitung auf Pr"ufungen:
\begin{enumerate}
 \item Durcharbeiten des Skripts
 \item Entspannungstechniken:
 \begin {itemize}
  \item autogenes Training
  \item Yoga
  \item Progressiv
 \end {itemize}
  \item Lerngrppen bilden
\end{enumerate}
Entspannungstechniken:
\begin {itemize}
 \item autogenes Training
 \item Yoga
 \item Progressiv
\end {itemize}

\subsection{Fu"snoten}
Fu"snote\footnote[1]{eine Fu"snote}

\subsection{Anf"uhrungsstriche}
\glqq Beginn des Zitats \grqq{} und Text geht weiter.\\
\glq Contract\grq - Verfahren

\subsection{Absatzformatierungen}
Vermerk und Benutzung in Pr"aambel.\\
Standardma"se: \\
$\backslash$parskip 0ex \\
$\backslash$parindent 1em \\
$\backslash$noindent (auch m"oglich)\\

\subsection{Horizontale Abst"ande}
Wort\negthickspace Wort\\
Wort\negmedspace Wort\\
Wort\negthinspace Wort\\
Wort\thinspace Wort\\
Wort\medspace Wort\\
Wort Wort\\
Wort\thickspace Wort\\
Wort\quad Wort\\
Wort\qquad Wort\\
0123\,4567\,8910\\
z.\,B.\\
B.\,Sc.\\

\subsection{Verticale Abst"ande}
Wort
\vspace{1ex}
Wort

\subsection{Trennstriche}
Trennstrich: Contracting-Verfahren\\
Bindestrich: Kraft-- und Saftlos\\
Streckenstrich: S.~32--46\\
Gedankenstrich: Meine Darstellung, dies wird auch von \textsc{Schulz} (1999) best"atigt -- bewei"st, dass \ldots

\subsection{Ligaturen}
bei \glqq ff\grqq, \glqq fi\grqq, \glqq fl\grqq \\
Unterdr"ucken durch, z.\,B.: \\
\glqq Auf"|lage\grqq

\subsection{Silbentrennung}
Silbentrennung:\\
xxxxxxxxxxxxxxxxxxxxxxxxxxxxxxxxxxxxxxxxxxxxxxxxxxxxxxxxxx
Wach\-stube\\
xxxxxxxxxxxxxxxxxxxxxxxxxxxxxxxxxxxxxxxxxxxxxxxxxxxxxxxxxx
Wachs\-tube\\
xxxxxxxxxxxxxxxxxxxxxxxxxxxxxxxxxxxxxxxxxxxxxxxxxxxxxxxxxx
Stau\-becken\\
xxxxxxxxxxxxxxxxxxxxxxxxxxxxxxxxxxxxxxxxxxxxxxxxxxxxxxxxxx
Staub\-ecken\\
xxxxxxxxxxxxxxxxxxxxxxxxxxxxxxxxxxxxxxxxxxxxxxxxxxxxxxxxxx
Klein\-eng\-lis\\
xxxxxxxxxxxxxxxxxxxxxxxxxxxxxxxxxxxxxxxxxxxxxxxxxxxxxxxxxx
\mbox{UNESCO}\\

\subsection{Querverweise}\label{Querverweise}
Im Abschnitt~\ref{Querverweise} geht es um Querverweise.
Dies ist der Text, auf den ich verweisen\label{marke S.117} m"ochte.\\
Wie schon bereits auf S.~\pageref{marke S.117}, beschrieben wurde
Keine Sonderzeichen in den Markentext.


\subsection{Randnotizen}
Randnotizen werden mit $\backslash$marginpar\{\} 
\marginpar{\tiny Tipp f"ur Klausur: Skript ausdrucken und durcharbeiten}
gesetzt.

\subsection{Beibehaltung der ursprünglichen Formatierung}
\verb:Originalformatierung (Schreibmaschine):
\begin{verbatim}
ein gro"ser Textbereich, \\
der so gestaltet ist, wie \\
das Original, \\
z.\,B. \LaTeX-Befehle
\end{verbatim}

\subsection{Kommentare}
Kommentare werden mit \% gesetzt. \\
oder: \\
$\backslash$begin\{comment\} \\
Text \\
$\backslash$end\{comment\} \\

\subsection{Sonderzeichen}
\subsubsection{Befehlszeichen}
\{ \\
\} \\
\# \\
\& \\
\_\\ %-
\% \\
\$ \\
\textbackslash \\ %\
\textasciitilde \\ %~

\subsubsection{Akzente}
%für alle Buchstaben: 
citt\'{a}capitale \\%aigu 
st\`{e}rile \\%grave
\^{o} \\%circonflex
fran\c{c}ois \\%c cedille
\~{n} \\
\v{s} \\
{\L} \\
%für skandinavische Sprachen:
{\aa} \\
{\o} \\
{\ae} \\
{\oe} \\
%französisch:
Lo\'{\i}r

\subsubsection{Textsymbole}
\S \\
\P \\
\texteuro \\
\officialeuro \\
\euro \\
\textyen \\
\pounds \\

\newpage
\section{Objekte}
1) feststehendes (nicht verschiebbares) Objekte: \\
  a) Formeln \\
  b) Tabellen \\
2) Gleitende (verschiebbare) Objekte: \\
  a) Formeln \\
  b) Tabellen \\
  c) Bilder / Grafiken \\
  
Entscheiden:

\subsection{feststehende Objekte}
\subsubsection{Tabellen}
%(1.)minimalistisch, zentriert, ohne Linien
\begin{tabular}{cc} %alternativ: l, r
 Ordnungsbegriffe &
 Schl"usselkompetenz, ENV, \LaTeX \\
 Zielgruppe &
 StudentInnen der Elektrotechnik \\
 Termine & Freitag, jeweils 6--9 Uhr \\
\end{tabular}

%(2.) wie (1.) mit vertikalen Linien 
\begin{tabular}{|c|c|} %alternativ: l, r
 Ordnungsbegriffe &
 Schl"usselkompetenz, ENV, \LaTeX \\
 Zielgruppe &
 StudentInnen der Elektrotechnik \\
 Termine & Freitag, jeweils 6--9 Uhr \\
\end{tabular}

%(3.) wie (2.) mit horizontalen Linien
\begin{tabular}{|c|c|} %alternativ: l, r
 \hline
 Ordnungsbegriffe &
 Schl"usselkompetenz, ENV, \LaTeX \\
 Zielgruppe &
 StudentInnen der Elektrotechnik \\
 Termine & Freitag, jeweils 6--9 Uhr \\
 \hline
\end{tabular}

%(4.) wie (3.) Tabelle zentriert
\begin{center}
 \begin{tabular}{|c|c|} %alternativ: l, r
 \hline
 Ordnungsbegriffe &
 Schl"usselkompetenz, ENV, \LaTeX \\
 Zielgruppe &
 StudentInnen der Elektrotechnik \\
 Termine & Freitag, jeweils 6--9 Uhr \\
 \hline
\end{tabular}
\end{center}
\vspace*{2ex}

%(5.) wie (4.) mit Doppellinie
\begin{center}
 \begin{tabular}{|c|c|} %alternativ: l, r
 \hline
 Ordnungsbegriffe &
 Schl"usselkompetenz, ENV, \LaTeX \\
 \hhline{|=|=|} %oder {|=+=|}
 Zielgruppe &
 StudentInnen der Elektrotechnik \\
 Termine & Freitag, jeweils 6--9 Uhr \tabularnewline[3ex]
 \hline
\end{tabular}
\end{center}
\vspace*{2ex}

%(6.) wie (5.) mit Leerzeilen
\begin{center}
 \begin{tabular}{|c|c|} %alternativ: l, r
 \hline
 &
 \tabularnewline[1ex]
 Ordnungsbegriffe &
 Schl"usselkompetenz, ENV, \LaTeX \\
 \hhline{|=|=|} %oder {|=+=|}
  &
 \tabularnewline[1ex]
 Zielgruppe &
 StudentInnen der Elektrotechnik \\
  &
 \tabularnewline[1ex]
 Termine & Freitag, jeweils 6--9 Uhr \tabularnewline[3ex]
 \hline
\end{tabular}
\end{center}
\vspace*{2ex}

\subsubsection{feststehende Formeln}
Die Formel $a = b\textsuperscript{3} + c\textsuperscript{2}$ wird grafisch abgesetzt vom Text.\\
Mehr abgesetzt:
\begin{math}
 a = b\textsuperscript{3} + c\textsuperscript{2}
\end{math}
Vom Text wird diese Formel, abgekürzt
\(
a = b\textsuperscript{3} + c\textsuperscript{2}
\)
\subsection{Formelzeichen}
Integrale:
\begin{math}
 \int^b_a f(x)\,dx 
\end{math}

Summenzeichen:
\begin{math}
 \sum_{i=1}^n a_x +2z 
\end{math}

Grenzwerte:
\begin{math}
\lim_{x \rightarrow \infty} x=0
\end{math}

\subsection{Gleitende Objekte}
Formeln\\
Tabellen\\
Grafiken\\

\subsubsection{gleitende Tabellen}
\begin{table}[ht!] %alternativ b (=bottom)
\centering
\caption[Kurzform]{Langform ... Text Text Text}
\label{Tab.1}
\vspace*{2ex}
  \begin{tabular}{|c|c|} %alternativ: l, r
 \hline
 Ordnungsbegriffe &
 Schl"usselkompetenz, ENV, \LaTeX \\
 \hhline{|=|=|} %oder {|=+=|}
 Zielgruppe &
 StudentInnen der Elektrotechnik \\
 Termine & Freitag, jeweils 6--9 Uhr \tabularnewline[3ex]
 \hline
\end{tabular}
\end{table}

\subsubsection{Gleitende Formeln}
nummeriert | nicht nummeriert \\
einzeilig | mehrzeilig \\
Eine einzeilge und nummerierte Formel:
\begin{equation}
 a = b^3 + c^2
\end{equation}

Eine nicht nummerierte und einzeilige gleitende Formel:
\begin{displaymath}
 a = b^3 + c^2
\end{displaymath}

Nummerierte mehrzeilige Formeln: %mit & Basteln
\begin{align}
 f(x)& =\cos x \\
 f'(x) &=& \sin x \\
 \int_{0}^1 f(x)\,dx& = &\tan b 
\end{align}

Nicht nummerierte mehrzeilige Formeln:
\begin{align*}
f(y)& = &\tan c \\
f'(2)& = &a\textsuperscript{2c} \\
\end{align*}

Zeilenumbruch:
\begin{equation}
 a = b + c + d + e + f + g + h + i + j + h + i + j + k + l + m + n + o + p + q + r + s + t + u + v + w + x + y + z + z + z + z + z + z + z + z + z + z + z + z + z
\end{equation}

\begin{align}
 \sin x = 2 - (2^2\cdot (a+b)) + (2^5 \cdot (b + c)) - \nonumber\\
 (2^6 \cdot (d+e))
\end{align}

Bruchstriche:
\begin{equation}
\label{Gleichung.1b}
 A = \frac{Z"ahler}{Nenner}
\end{equation}
In der Gleichung~\ref{Gleichung.1b} wird bewiesen, dass ... \\
In der Gleichung~\eqref{Gleichung.1b} wird bewiesen, dass ... \\

\subsection{Grafiken}

\begin{comment}
 
\begin{figure}
 \includegraphics{eigene-grafik}
\end{figure}

\end{comment}

\section{Bibliografie und Index}
\subsection{Index}
Die Formel \index{Formel} $y=mx + b$ wird umgeformt zu
\begin{displaymath}
 x = \frac{y - b}{m}
\end{displaymath}
Diese Masterarbeeit enth"alt viele Formeln 
\index{Formel}, also viel Mathematik \\
\index{Mathematik!Formel}
Wegen ihrer L"ange hat sie viele "Uberschriften \\
\index{Ueberschrift@Überschrift}
"Uberf"uhrungn\index{Ueberfuehrung@Überführung} \\
\LaTeX - Befehle \index{LaTeX@\textbackslash LaTeX}
die Messtelle \glqq Ing. Schule\grqq \\
\index{Ing.Schule@\glqq Ing.Schule\grqq}

%\printindex %hier steht der Index

\subsection{Literaturverzeichnis}

1. Import aus externen Datenbanken $\rightarrow$ ASCII \\
2. BibTex \\
3. in tex-Datei Literaturdaten \\

\cite{Stenger} beschrieb Gedächtsnistraining
\cite[Seite 53]{edmonson 1984} gab Hinweise zum Weg einer Veröffentlichung bis zum gedruckten Erzeugnis.

\subsection{Literatur Bibliografie}
\begin{thebibliography}{99}
 \bibitem{Stenger}
 \textsc{Stenger}, C.2013;
 \emph{Ged"achtnistraining mit der Jugendweltmeisterin.} 6.Auflage, M"unchen Heyne, 232 Seiten.
 \bibitem[TU]{Turk 1991}
 \textsc{Turk}, C.1991:
 \emph{Effective speaking: communication in science.} 1.Auflage,London: Spon, IX+270 Seiten.
 \bibitem{edmonson 1984}
 \textsc{Edmonson}, Y.H.1984:
 \emph{The natural history of a manscript.} .-- Limnology \& Oceanolography, \textbf{29} (5): 1145--1148.
\end{thebibliography}










\printindex

\end{document}



Ende der Musterdatei
