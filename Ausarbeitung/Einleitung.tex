!TEX root = Praktikumsarbeit.tex  %zur fehlerfreien Compilierung und Einbettung

\section{Einleitungs}
Einleitungstext, worum geht es hier bei diesem Projekt?

\subsection{Motivation} 


Die\marginpar{\tiny Automaisierung und warum?} Motivation sich mit dem Thema der Automatisierung im Land und Gartenbau zu beschäftigen, kann wohl jeder verstehen, der häufiger mal eine Dienstreise antreten muss. Kommt man nach 2 Wochen von einem Anstrengenden nach Hause sind die Pflanzen im Garten womöglich zusammen mit dem Unkraut vertrocknet. Die Zeitarbeitskraft die ein Garten bindet ist gerade im Stadium des Pflanzenwachstums immens.
\newline
Die bisherigen Ansätze zur Automatisierung der Landwirtschaft beschäftigten sich zumeist nicht damit, die Zeit des Arbeiters auf dem Feld zu reduzieren, sondern lediglich die zu bearbeitende Fläche zu erhöhen. So ermöglicht der Einsatz des Pflugs gegenüber dem Spaten ein vielfaches der Fläche in einem Arbeitstag zu bearbeiten. 
\newline 
Die menschliche Arbeitskraft hat für uns dabei noch mehr Nachteile, als den bloßen Faktor Mensch. Ein Mensch kann nicht für gleichbleibende Feuchtigkeit im Boden sorgen, sondern sorgt immer für eine gleichbleibende Nachgießmenge. Wir wollen also in der Zeit der knappen Wasser und Ackerflächen und anschauen, ob es in diesem Bereich Optimierungspotentiale gibt und wie man diese mit einer Maschine nutzbar machen kann. Damit können wir einen Beitrag leisten, den Wasserverbrauch von frischen Gemüse zu reduzieren. Dabei gilt es, dass unsere Maschine Daten aufnehmen soll um damit Auswertungen die äußeren Bedingungen der Pflanzen fahren zu können. 

Dabei geht es also darum, die wiederkehrenden Tätigkeiten der Gartenarbeit zu verringern und die Auswertbarkeit weiter in den Mittelpunkt zu stellen. Ziel dabei sollen die Verrinerung von Ressourcen und Arbeitstätigkeit je Gemüseeinheit werden.
\newline
Dabei wollen wir einen Pflanzenumgang der möglichst ohne das Einbringen von Pestizieden oder Fungizieden auskommt.

\subsection{aktueller Stand}
kurze Bestandsmeldung der agrarwirtschaftlichen Methoden

\subsection{Projektbeschreibung}
auf welche Aspekte aus der Bestandsmeldung wird hier weiterführend eingegangen
