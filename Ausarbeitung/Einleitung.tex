!TEX root = Praktikumsarbeit.tex  %zur fehlerfreien Compilierung und Einbettung

\section{Einleitungs}
Einleitungstext, worum geht es hier bei diesem Projekt?

\subsection{Motivation} 

Die Motivation sich mit dem Thema der Automatisierung im Land- und Gartenbau zu beschäftigen, hat mehrere Ursachen. Die Gedanken eines Hobby-Gärtners, der bereits einmal vergessen hatte zu gießen, können genauso von der Automatisierung des Gartens und des Gießens träumen, wie ein Feldarbeiter, der seine zu bearbeitende Fläche vergrößern will, ohne dabei an Gemüsequalität einzubüßen.
\newline
Dabei gilt es zu beachten, dass die bisherigen Ansätze zur Automatisierung der Landwirtschaft sich zumeist nicht damit beschäftigten, die Zeit des Arbeiters auf dem Feld zu reduzieren, sondern lediglich die zu bearbeitende Fläche zu erhöhen. So ermöglicht der Einsatz des Pflugs gegenüber dem Spaten ein Vielfaches der Fläche in einem Arbeitstag zu bearbeiten. Wir haben dabei eine andere Motivation. Wir wollen die Entscheidung zu Gießen und wann es nötig ist, automatisieren. Ziel dieser Arbeit ist es zwar unter anderem auch automatisiert zu gießen, aber eben auch, dass die Maschine mit den Fähigkeiten ausgestattet ist, die Entscheidung zum Gießen selbst zu fällen.
\newline 
Menschliche Arbeitskraft gilt es für uns dabei auf ein Minimum zu reduzieren. Das hat dabei für uns mehrere Aspekte. Einem Menschen fällt es schwer Zustandsgrößen in einem Feld zu erreichen. Bodenfeuchtigkeit kann er, statt mathematisch, nur ungefähr bestimmen. Dadurch wird unter Umständen zu viel Wasser ausgeschenkt, was für ein schlechteres Pflanzenwachstum sorgen kann. Wir wollen uns, in der Zeit der knappen Wasserressource und Ackerflächen, anschauen, ob es in diesem Bereich Optimierungspotentiale gibt und wie man diese mit einer Maschine nutzbar machen kann. Damit können wir einen Beitrag leisten, den Wasserverbrauch von frischem Gemüse zu reduzieren. Dabei gilt es, dass unsere Maschine Daten aufnehmen soll, um damit Auswertungen über die äußeren Bedingungen der Pflanzen fahren zu können und deren Auswirkungen beschreiben zu können. Wir wollen die Maschine prinzipiell über Customizing-Größen in die Lage versetzt wird, selbst Entscheidungen z.B. zum Gießen und Harken zu fällen. 

Dabei geht es also darum, die wiederkehrenden Tätigkeiten der Gartenarbeit zu verringern und die Auswertbarkeit weiter in den Mittelpunkt zu stellen. Das große Ziel dieses Jahrhunderts ist, mit gesammelten Daten, die Potentiale in der Pflanzenzucht aufzuzeigen und mit der Möglichkeit zum maschinellen Lernen die landwirtschaftlichen Erträge, ohne zusätzliche Zugabe von Chemikalien, zu steigern. Ziele dabei sollen die Verringerung von Ressourcen, Arbeitstätigkeit und Umwelteinflussnahme je Gemüseeinheit werden. 
\newline
Die Idee dazu kam während der Vorlesung Digitale Systeme in der es um die Grundidee ging eine Zimmerpflanze möglichst effizient zu bewässern. Nachdem dies ein besseres Wachstum bei einem Kaffeebaum zur Folge hatte, kam der Gedanke auf, mit Methoden der Automatisierung ein System aufzubauen, dass eine automatisierte Pflanzenpflege betreiben kann. Daher kommt die Motivation zu diesem Produkt.
\subsection{aktueller Stand}
kurze Bestandsmeldung der agrarwirtschaftlichen Methoden

\subsection{Projektbeschreibung}
auf welche Aspekte aus der Bestandsmeldung wird hier weiterführend eingegangen
